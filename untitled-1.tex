\documentclass[10pt, oneside]{article}

\usepackage[T1]{fontenc}
\usepackage[utf8]{inputenc}
\usepackage{polski}
\usepackage{indentfirst}
\usepackage{caption}
\usepackage{float}
\usepackage{tikz}
\usepackage{polski}
\usepackage{fancyhdr}
\usepackage{lastpage}
\usepackage{tcolorbox}
\usepackage{graphicx}

\title{Specyfikacja funkcjonalna programu WireWorld}
\author{Danuta Stawiarz, Katarzyna Stankiewicz}
\date{3 maja 2019 r.}

\pagestyle{fancy}
\fancyhf{} 
\lhead{}
\rhead{} 
	
\rfoot{
\begin{center} Strona \thepage \hspace{1pt} z \pageref{LastPage}
\end{center}
}
%____________________________________________________________________

\begin{document}
\maketitle
\tableofcontents
\newpage	

\section{Cel projektu}

Celem projektu jest napisanie programu WireWorld. Program zaimplementowany zostanie w języku java z użyciem biblioteki graficznej JavaFX. Po ukończeniu program będzie w stanie przeprowadzać proste symulacje, poprzez generowanie kolejnych planszy od ustawionej na początku przez użytkownika, podanej jako plik wejściowy lub losową w przypadku braku działań użytkownika. Ponadto możliwa będzie również interakcja użytkownika z programem poprzez interfejs graficzny i odpowiednie przyciski. Użytkownik będzie miał możliwość edycji danej planszy. 

\section{Opis zasad programu}

Automat komórkowy WireWorld ma za zadanie wykonać symulację, w której przekształca podaną planszę na podstawie kilku podstawowych zasad.

 Komórka może znajdować się w jednym z czterech stanów:

\begin{enumerate}
\item Pusta,
\item Głowa elektronu,
\item Ogon elektrony,
\item Przewodnik.
\end{enumerate}

W zaimplementowanym automacie przyjmuje się następujące kolory stanów:
\begin{itemize}
\item czarny - pusta
\item niebieski - głowa elektronu
\item czerwony - ogon elektronu
\item biały - przewodnik
\end{itemize}


Kolejne generacje budowane są z wykorzystaniem zestawu pięciu zasad:
\begin{enumerate}
\item Komórka pozostaje Pusta, jeśli była Pusta.
\item Komórka staje się Ogonem elektronu, jeśli była Głową elektronu.
\item Komórka staje się Przewodnikiem, jeśli była Ogonem elektronu.
\item Komórka staje się Głową elektronu tylko wtedy, gdy dokładnie 1 lub 2 sąsiadujące komórki są Głowami Elektronu.
\item Komórka staje się Przewodnikiem w każdym innym wypadku.
\end{enumerate}

W WireWorld stosuje się sąsiedztwo Moore'a.
\section{Wygląd interfejsu użytkownika}

\section{Opis działania programu}
W momencie uruchomienia programu pojawia się menu umożliwiające użytkownikowi wybór symulacji, którą chce przeprowadzić. W zależności od wyboru WireWorld lub Game of Life wyświetlone zostanie okno pozwalające na przeprowadzenie symulacji adekwatnej do wybranej opcji.

\subsection{Ustawienia początkowe}
Na wejściu użytkownik może podać plik zawierający dane do wczytania planszy lub podać wymiary planszy która ma zostać wczytana. W podania wymiarów użytkownik zostanie poproszony również o wybranie stanu w jakim ma zostać wczytana plansza. 
\begin{itemize}
\item wypełnij losowo - plansza ma zostać wypełniona losowo
\item pusta plansza - wszystkie komórki są martwe (dla  Game of Life) lub puste (dla WireWorld)
\end{itemize}

\subsection{Zarządzanie planszą}
Obok planszy będą wyświetlane następujące opcje:
\begin{itemize}
\item wczytaj plik -  za pomocą tego przycisku użytkownik będzie mógł wczytać wybrany przez siebie plik
\item start/stop - umożliwia wstrzymanie lub uruchomienie symulacji
\item wymiary - dwa okienka: x - szerokość planszy y - wysokość planszy
\item generacje - wyświetla ile generacji upłynęło od czasu działania programu, \item strzałki obok generacji - umożliwiają manualne przejście do kolejnej generacji lub powrót do poprzedniej (aktywne tylko w przypadku zatrzymanej symulacji)
\item zapisz - umożliwia zapisanie planszy do pliku

Obok planszy będzie się znajdowało również okno pozwalające na edycję aktualnej planszy. Będzie ono aktywne tylko w przypadku zatrzymanej symulacji.
W przypadku zatrzymanej symulacji istnieje możliwość wyboru komórki na planszy i zmianę jej stanu.

\end{itemize}
\subsection{Przykładowy plik wejściowy}
W obu wariantach gry pliki wejściowe muszą zawierać informację o wymiarach planszy oraz stanach poszczególnych komórek w niej umueszczonychl.
\subsubsection{WireWorld}
W tym wariancie gry komórki mogą przyjmować jeden z 4 stanów. W pliku tekstowym przyjmują one kolejno postać cyfr:
\begin{itemize}
\item 1- Komórka pusta
\item 2- Głowa elektronu
\item 3- Ogon elektronu
\item 4- Przewodnik
\end{itemize}
Przykładowy plik może wyglądać następująco:
\begin {verbatim}
4 4
1 2 0 1
3 4 1 0
0 1 1 2
3 4 0 0 
\end{verbatim}
Wymiary planszy są podane w pierwszej linijce, a w kolejnych umieszczono dane o stanach komórek.

\subsubsection{Game of Life}
W tym wariancie gry komórki mogą przyjmować jeden z 2 stanów. W pliku tekstowym przyjmują one kolejno postać cyfr:
\begin{itemize}
\item 0- Komórka  żywa
\item 1- Komórka martwa
\end{itemize}
Przykładowy plik może wyglądać następująco:
\begin {verbatim}
4 4
1 0 0 1
1 1 1 0
0 1 1 0
1 0 0 0 
\end{verbatim}
Wymiary planszy są podane w pierwszej linijce, a w kolejnych umieszczono dane o stanach komórek.






\section{Wyniki działania programu}
Wynikiem działania programu jest plik tekstowy zawierający dane o obrazie ostatniej generacji komórek. Plik ten zostaje zapisany pod podanym przez użytkownika adresem.  Na bieżąco użytkownik może też obserwować wygląd planszy wraz ze stanami poszczególnych komórek online.

\section{Komunikaty błędów}
\subsection{Błąd pliku wejściowego}
W przypadku niepowodzenia wczytania pliku zostanie wyświetlony komunikat "Nie udało się wczytać pliku".
Jeśli plik istnieje, ale zawiera błędy, komunikat będzie zawierał informację w której linii pojawił się błąd. 
"Nieprawidłowa wartość w linii x, nierozpoznany znak: <błędny znak>"
\subsection{Błędy w obsłudze planszy}
 Jeśli użytkownik spróbuje podać nieprawidłowy rozmiar planszy, program wyświetli komunikat "Podano nieprawidłowy rozmiar planszy"
\subsection{Błąd zapisu do pliku}
W przypadku niepowodzeniazapisu pliku zostanie wyświetlony komunikat "Nie udało się zapisać do pliku". Program poprosi uzytkownika o wskazanie innego adresu zapisu.






\end{document}